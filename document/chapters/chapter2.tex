\chapter{State Of The Art}
\label{chapter:SOTA}

\begin{introduction}
A short description of the chapter.

A memorable quote can also be used.
\end{introduction}

The following chapter aims to expose the discovery work carried out during the preparation of this thesis.

\section{Serverless}
The last decade has seen the expoential growth of cloud computing. 
Its expanding set of tools and solutions, coupled with increased accessibility both economically and in the demanded levels of expertise required for interactions, have made it relevant for an immense variety of use cases.\par

Given its success, newer frameworks and paradigms have been idealized and developed, among which one of the most interesting has been Serverless Computing. 
This new technology started showing up in the market around the early to mid 2010's with limited capabilities, however as it started growing, it's potential started being aparent.\par

In many ways, the jump from Cloud to Serverless computing can be seen as mirroring the jump from low to high-level programming languages \cite{DBLP:journals/corr/abs-1902-03383}. 
Whereas an Assembly programmer is required to have a deeper knowledge of the hardware architecture by being given tasks such as selecting registers for his variables and managing the CPU memory, in traditional cloud computing the architect is also required to identify and provision the needed resources among many other computational duties. 
On the other hand, serverless computing abstracts many of these tasks with the biggest parallel in regards to high-level languages being the abstracted need to allocate and manage resources (server resources on the serverless side and memory resources on the programming side).\par

In a nutshell, serverless is an architectural pattern where the resources are automatically managed and the server is abstracted away from the user, although, in contradiction with its name, the program still runs in a server \cite{Buyya2019}.

\subsection{Achievements and Shortcomings of Serverless}
The march towards higher levels of abstraction is a common and expected behaviour of software systems, however it remains a gradual process where iterations of inovations are needed to fully encompass the new way of doing things.\par

With this in mind, we will analyze the current capabilities of Serverless frameworks along with the benefits they add to the cloud landscape followed by their shortcomings and points of needed improvement to reach the full potential of this novel pattern.\par

To define Serverless by the sum of its intrinsic features we can adapt the description given by \citet{10.1145/3508360}

